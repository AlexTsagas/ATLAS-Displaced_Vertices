\section{Εισαγωγή}

\begin{frame}{Βασικοί Ορισμοί}
	
	\begin{itemize}
		\item Τα \textbf{Interaction Points} (\textbf{IPs}) είναι τα σημεία κατά μήκος του LHC όπου οι δέσμες πρωτονίων διασταυρώνονται και συγκρούονται.
		\item Ένα \textbf{long-lived particle} αποτελεί ένα σωμαδίο με σχετικά μεγάλο χρόνο ζωής το οποίο προβλέπεται από θεωρίες που επεκτείνουν το Standard Model.
		\item Μία \textbf{Primary Vertex} (\textbf{PV}) είναι το σημείο από όπου προέρχονται δύο ή περισσότερες τροχιές και συμπίπτει με το IP.
		\item Μία \textbf{Displaced Vertex} (\textbf{DV}) είναι το σημείο όπου δύο ή περισσότερες τροχιές συγκλίνουν και βρίσκεται σε σημαντική απόσταση από το IP. 
	\end{itemize}
\end{frame}

\begin{frame}{Ορισμοί Αλγορίθμου}
	
	\begin{itemize}
		\item Τα \dvtruebf\ αναφέρονται στα πραγματικά DV που συναντώνται στα γεγονότα.
		\item Τα \dvrecobf\ αναφέρονται στα DV που υπολογίζει το πρόγραμμα.
		\item \textbf{Σφάλμα}/\textbf{Error} ονομάζεται η απόσταση μεταξύ του \dvtrue\ και του αντίστοιχου \dvreco.
		\item \textbf{Απόσταση μεταξύ δύο ευθειών} ορίζουμε το ελάχιστο της απόστασης ενός σημείου της πρώτης από τη δεύτερη.
	\end{itemize}
\end{frame}

\begin{frame}{Σκοπός}

	\begin{enumerate}
		\item Ανάπτυξη αλγορίθμου που αναζητά και αναγνωρίζει τα \dvtrue\ που υπάρχουν σε πολλαπλά γεγονότα.	
		%
		\begin{itemize}
			\item \textbf{Δείκτες:}
				\setbeamertemplate{enumerate items}[square]
				\begin{enumerate}
					\item \textbf{Απόδοση:} Ο λόγος των \dvtrue\ που αντιστοιχίζονται σε κάποιο \dvreco\ δια το συνολικό αριθμό των \dvtrue.
					\item \textbf{Καθαρότητα:} Ο λόγος των \dvreco\ που αντιστοιχίζονται σε κάποιο \dvtrue\ δια το συνολικό αριθμό των \dvreco.
				\end{enumerate}
			\item \textbf{Δεδομένα από Ιστογράμματα:}	
				\setbeamertemplate{enumerate items}[square]
				\begin{enumerate}
					\item \textbf{Ακρίβεια:} Ο λόγος του αριθμού των \dvreco\ με σφάλμα μικρότερο από ένα όριο προς το συνολικό αριθμό των \dvreco.
					\item \textbf{Αποτελεσματικότητα:} Σύγκριση αριθμού \dvreco\ και \dvtrue.
				\end{enumerate}
		\end{itemize}
		%
		\item Σύγκριση αποτελεσμάτων με αυτά που προκύπτουν από ανθρώπινη είσοδο.
	\end{enumerate}

\end{frame}

\begin{frame}{Χαρακτηριστικά Γεγονότων}
	
	Τα γεγονότα υπό επεξεργασία έχουν τα εξής χαρακτηριστικά:
	%
	\vspace{4mm}
	%
	\begin{itemize}
		\item Ο αριθμός τους είναι $4300$.
		\item Όλες οι τροχιές που σχετίζονται με PV έχουν αφαιρεθεί.
		\item Όλα τα γεγονότα περιέχουν τουλάχιστον ένα DV.
		\item Τα γεγονότα περιέχουν προσομοιωμένα δεδομένα.
		\item Αριθμοί \dvtrue\,:
			\begin{itemize}
				\item Συνολικά: $5247$.
				\item Γεγονότα με ένα \dvtrue\,: $3358$.
				\item Γεγονότα με δύο \dvtrue\,: $1868$.
				\item Γενοτότα με τρία \dvtrue\,: $21$.
			\end{itemize}
	\end{itemize}
	
\end{frame}
